\documentclass[a4paper,12pt]{article}
\usepackage[T2A]{fontenc}
\usepackage[utf8]{inputenc}
\usepackage[english,russian]{babel}
\usepackage{indentfirst}
\usepackage{amssymb}
\usepackage{amsfonts}
\usepackage{amsmath}
\usepackage{mathtext}
\usepackage{cite}
\usepackage{enumerate}
\usepackage{float}
\usepackage[top=1.5cm,bottom=1.5cm,left=2.5cm,right=1.5cm]{geometry}
\usepackage[unicode]{hyperref}
\usepackage{graphicx}
\usepackage{color}
\usepackage[colorinlistoftodos]{todonotes}
\usepackage[format=hang, labelsep=period, margin=7pt, figurename=Рис.]{caption}
\usepackage{listings}
\usepackage{subcaption}
\usepackage[onehalfspacing, nodisplayskipstretch]{setspace}
\usepackage{hyperref}

\definecolor{dkgreen}{rgb}{0,0.6,0}
\definecolor{gray}{rgb}{0.5,0.5,0.5}
\definecolor{mauve}{rgb}{0.58,0,0.82}

\numberwithin{equation}{section}
\setcounter{secnumdepth}{3}
\setcounter{tocdepth}{3}

\begin{document}
\section{Постановка задачи}

\subsection{Цели и задачи модификации}	
	\begin{itemize}
		\item{обеспечить по итогам текущей модернизации кода удобство интегрирования новых 	реологических моделей МСС, новых численных алгоритмов, выполняющих расчёт этих моделей;}
		\item{сделать код более структурированным, более понятным и простым для ознакомления и будующих расширений.}
	\end{itemize}
	
\subsection{Постановка задачи}

	На данный момент задача в самой общей постановке представляет собой систему \textit{N} квазилинейных гиперболических уравнений в частных производных:
\begin{equation}
	\label{main_equation}
	\frac{\partial\vec{u}}{\partial{t}}+\mathbf{A}_x\frac{\partial\vec{u}}{\partial{x}}+
	\mathbf{A}_y\frac{\partial\vec{u}}{\partial{y}}+
	\mathbf{A}_z\frac{\partial\vec{u}}{\partial{z}}=\vec{f},
\end{equation}
	где $\mathbf{A}_x = \mathbf{A}_x(\vec{u}, \vec{r}, t)$, $\mathbf{A}_y = \mathbf{A}_y(\vec{u}, \vec{r}, t)$, $\mathbf{A}_z = \mathbf{A}_z(\vec{u}, \vec{r}, t)$, $\vec{f} = \vec{f}(\vec{u}, \vec{r}, t)$.
	
	Также движение сетки описывается уравнением:
\begin{equation}
	\label{mesh_movement_equation}
	\frac{\partial\vec{r}}{\partial{t}}=\vec{v}.
\end{equation}	
	
	Кинетика разрушений в общем случае описывается системой эволюционных уравнений:
\begin{equation}
	\label{failure_equations}
	\frac{\partial{\vec{\chi}}}{\partial{t}}=\vec{F}(\vec{r}, t, \vec{u}, \vec{\chi}),
\end{equation}		
	где $\vec{\chi}$ -- так называемые внутренние параметры, характеризующие внутреннюю структуру материала (пористость, размер пор, повреждённость, параметр упрочнения и пр.).
	
	Уравнение \eqref{mesh_movement_equation} подразумевает замену дифференциального оператора разностным нужного порядка.
	
	Уравнения \eqref{failure_equations} могут иметь различный вид. Пока у нас простейшие модели -- решение этих(этого) уравнения не составит труда. Дальше будем думать.
	Вместо этих уравнений могут быть различные критерии.
	

\section{Общий подход}

\subsection{Принципиальная архитектура}

Код модульный. <<Большие>> сущности (модель, метод, сетка -- см.ниже) являются чисто описательными классами, содержащими соответствующие базовые <<кирпичики>> (корреторы, сеттеры, интерполяторы и т.д.).

\subsection{Что задаётся в таске}

В таске задаются:
	\begin{itemize}
		\item{физическая модель для неразрушенного тела (\textbf{DeformationModel} по тексту ниже),}
		\item{физическая модель разрушения (\textbf{FailureModel} по тексту ниже),}
		\item{численный метод (\textbf{Solver} по тексту ниже),}
		\item{тип сетки (\textbf{Mesh} по тексту ниже),}
		\item{интерполятор (\textbf{Interpolator} по тексту ниже, зависит от метода и сетки, фактически определяет порядок по пространству),}
		\item{граничные и контактные условия (\textbf{Calculator} + область действия).}
	\end{itemize}

\section{Модель}
\subsection{DeformationModel}
	Для пущей структурированности и удобства имплементации различной физики предлагаю ввести класс \textbf{DeformationModel}.
	
	Переменная \textbf{DeformationModelName}.

	В этот класс предлагаю ввести следующие сущности:
	\begin{itemize}
		\item{\textbf{Material} material -- материал,}
		\item{Информация о требуемом типе \textbf{Node}'ов. Далее при создании сетки должны использоваться \textbf{Node}'ы этого типа. Вероятно, потребует шаблонизации класса сетки. Общая схема, вероятно, по аналогии текущим видом MeshLoader'ов -- модель создаёт сетку с нодами нужного типа, после чего Loader грузит геометрию,}
		\item{\textbf{MatrixSetter} matrixSetter -- заполняет матрицы текущей модели,}
		\item{\textbf{InGomogeniousSetter} inGomogeniousSetter -- заполняет правую часть уравнений,}
		\item{\textbf{Correctors} correctors -- набор корректоров, если модель их подразумевает (пример - идеальная пластика на базе корректора.}
	\end{itemize}

	Внутри всего указанного неявно скрывается размерность вектора $\vec{u}$.
	
	Варианты моделей:
	\begin{itemize}
		\item{Elasticity,}
		\item{ElasticityFiniteStrains,}
		\item{NonLinearElasticity,}
		\item{Plasticity,}
		\item{ThermoElasticity.}
	\end{itemize}
	
\subsection{FailureModel}

	\textbf{FailureModel} с хорошей точностью совпадает с текущей реализаций. Содержит набор критериев и корректоров. Критерии и корректоры логически включают в себя как уравнения, так и алгоритм их решения (не обращаются в общем случае к дополнительным внешним классам). 
	
	Переменная \textbf{FailureModelName}.
	Переменная \textbf{FailureModelType} -- = \textbf{discrete} \textit{or} = \textbf{continuum}.
	
	Примерные варианты реализаций(дочерние классы):
	\begin{itemize}
		\item{FailureDiscrete,}
		\begin{itemize}
			\item{MisesCriterion,}
			\item{MohrCoulombCriterion,}
			\item{HashinCriterion,}
			\item{TsaiHillCriterion,}
			\item{TsaiWuCriterion;}
		\end{itemize}
		\item{FailureContinuum,}
		\begin{itemize}
			\item{Failure1Order,}
			\item{Failure2Order;}
		\end{itemize}
	\end{itemize}
	
\section{Метод}
\subsection{GCMsolver}
	Родительский класс -- \textbf{GCMsolver}. Его предлагаю сделать обобщённым.
	
	Содержит переменные \textbf{spaceOrder}, \textbf{timeOrder}, \textbf{bufNodes}.
	А также \textbf{bufСoefficients} -- наклоны характеристик, которые уточняются в процессе решения.
	
	Далее, флаг \textbf{HomogeniousSystem}. Необходимы отдельно метод для однородных систем и отдельно для неоднородных систем.
	
	Далее, поскольку у нас сеточно-характеристический метод и больше никого (появится FEM будет FEMsolver) разумно определить сюда базовые кирпичи GCM'а.
	
	\begin{itemize}
		\item{\textbf{CrossPointFinder} сrossPointFinder - ищет точку пересечения характеристики,}
		\item{\textbf{ModelPtr} -- ссылка на текущую модель,}
		\begin{itemize}
			\item{\textbf{MatrixSetter} matrixSetter -- заполняет матрицы,}
			\item{\textbf{InGomogeniousSetter} inGomogeniousSetter -- заполняет правую часть уравнений,}
		\end{itemize}
		\item{\textbf{Decomposer} decomposer -- разлагает матрицы (солвер несёт с собой <<декомпозер по умолчанию>>, модель может (как именно?) сказать, что нужен другой декомпозер (пример -- у меня изотропное тело и линейная упругость, разложение матриц известно в явном виде)),}
		\item{\textbf{InGomogeniousSolver} ingomogeniousSolver -- используется для моделей с правой частью, решает уравнение на перенос инвариантов $\vec{\xi}$:}
		\begin{equation}
			\label{invariant_equation}
			\frac{\partial\vec{\xi}}{\partial{t}}=\mathbf{\Omega}\vec{F},
		\end{equation}
		\item{\textbf{Interpolator} interpolator -- интерполирует значения инвариантов в точке, зависит от типа \textbf{Mesh}'a (не всё ко всему применимо), полностью инкапсулирует логику получения нового значения (классический гибридный метод -- это фактически такой умный интерполятор, который использует то 1-ый, то 2-ой порядок), задаётся в таске (<<а давайте всё то же самое, но теперь вторым порядком>>),}
		\item{\textbf{MeshMover} meshMover -- двигает узлы, решая уравнение \eqref{mesh_movement_equation} нужным порядком,}
		\item{\textbf{FailureSolver} failureSolver -- отвечает за решение \eqref{failure_equations}, скорее всего, является просто методом, физически вызывая \textbf{FailureModel}, в которой критерии и корректоры содержат нужную математику целиком.}
	\end{itemize}
	
	Если кто знает TVD, ENO, аппроксимационную вязкость или любую вязкость -- милости просим. (В смысле, это будет отдельный солвер, возможно, завязанный на тип сетки, с которым умеет работать.)
	
	Основной метод \textbf{doNextTimeStep} -- выполняет следующий шаг по времени.

\subsection{Calculator'ы}
	Собственно:
	\begin{itemize}
		\item{VolumeCalculator,}
		\item{BorderCalculator,}
		\item{ContactCalculator.}
	\end{itemize}
	Примерно в том виде, какой есть. Не все применимы ко всем моделям.

\subsection{CrossPointFinder}
	Родительский класс. Производные классы -- реализации конкретного порядка и просто разные реализации.
	В зависимости от реализации получает набор нодов и реологических параметров
	
	Переменная \textbf{spaceOrder}.
	
	Флаг \textbf{linearCharateristics} -- если характеристики -- прямые линии, в соответствии с моделью тут следует использовать первый порядок, дающий точное положение. Такакя же ситуация как и с \textbf{HomogeneusSystem}.
	Тут два варианта: 
		\begin{itemize}
			\item{отдельно метод второго порядка для прямых характеристик, с указанием нужного метода в модели;}
			\item{На этапе инициализации \textbf{Model} программа понимает, что нужно использовать линейный \textbf{Finder} для метода второго порядка.}
		\end{itemize}

	Предполагаемые реализации:
		\begin{itemize}
			\item{CrossPointFinder1Order;}
			\item{CrossPointFinder2Order.}
		\end{itemize}

\subsection{Interpolator}
	Зависит от типа \textbf{Mesh}'a (не все со всеми могут работать).
	
	Неоходимы отдельные реализации классов:
	\begin{itemize}
		\item{Interpolator1Order;}
		\item{Interpolator2Order;}
		\begin{itemize}
				\item{Interpolator2Order;}
				\item{InterpolatorLimiter,}
				\begin{itemize}
					\item{InterpolatorLimiterMinMax,}
					\item{InterpolatorLimiterLinear.}
				\end{itemize}
		\end{itemize}
	\end{itemize}
	
\subsection{MatrixSetter}
	Получает \textbf{curNode} и заполняет матрицы $\mathbf{A}_x$, $\mathbf{A}_y$, $\mathbf{A}_z$ в соответствии с моделью.
	
\subsection{InGomogeniousSetter}
	Получает \textbf{curNode} и заполняет правую часть неоднородного уравнения $\vec{F}$ в соответствии с моделью.
	
\subsection{MeshMover}
	Переменная \textbf{timeOrder}.
	Решает уравнение \eqref{mesh_movement_equation} в соответствии с порядком:
	\begin{itemize}
		\item{MeshMover1Order;}
		\item{MeshMover2Order.}
	\end{itemize}	
	
	Получает \textbf{curNode} и двигает его в соответствии с решением.

\subsection{Decomposer}
	Реализации:
	\begin{itemize}
			\item{NumericalDecomposer;}
			\item{AnalyticalDecomposer,}
			\begin{itemize}
					\item{IsotropicDecomposer,}
					\item{GeneralAnalyticalDecomposer.}
			\end{itemize}
	\end{itemize}

\subsection{InGomogeniousSolver}
	Переменная \textbf{timeOrder}.
	
	Решаем уравнение на инварианты \eqref{invariant_equation}.
	
	Предполагаемые реализации:
		\begin{itemize}
			\item{InGomogeniousSolver1Order;}
			\item{InGomogeniousSolver2Order.}
		\end{itemize}


\section{Сетка}

Более-менее соответствует нынешнему виду. Должна учитывать: 
	\begin{itemize}
		\item{параллельность;}
		\item{потенциальную возможность перестройки в ходе счёта.}
	\end{itemize}

\end{document}
